\begin{abstract}
	Sketch is a probabilistic data structure used to record frequencies of items in a multiset.
	It can work with small memory usage and achieve high speed and accuracy.
	Therefore, sketches have been widely used in many applications, such as data stream processing, distributed datasets, natural language processing, and network traffic analysis.
	However, conventional sketches are not accurate enough for non-uniform datasets, which are common in practice.
	In this paper, we propose a novel technique, named \fname~(\aname), which can improve the accuracy of sketches for non-uniform datasets without extra costs.
	The key idea of our technique is to use different parameters, including width and counter size, for different arrays in a sketch.
	We applied our novel technique to several well-known sketches, including CM sketches and CU sketches.
	We also made mathematical analysis to prove that our technique is able to improve the accuracy of sketches.
	Extensive experimental results show that our \aname~has improved the accuracy of existing sketches by up to ...
	The source codes of our new technique are available on our homepage.
\end{abstract}